\documentclass{beamer}
%\usetheme[footline=infoline,headline=secheader]{UPB}
%\usetheme[footline=infoline,headline=structure]{UPB}
%\usetheme[headline=secheader]{UPB}
%\usetheme[headline=structure]{UPB}
%\usetheme[footline=infoline]{UPB}
%\usetheme{UPB} % defaults are footline=empty,headline=empty
%\usetheme{Antibes}
\usetheme{upb}
\usepackage[utf8]{inputenc}
\usepackage{hyperref}
\author[C.Robbert, P. Stilow]{Christoph Robbert, Peter Stilow}
\institute[Uni Paderborn]{Universität Paderborn}
\title[WLAN Security]{WLAN Security}
\begin{document}
\begin{frame}
\maketitle
\end{frame}
\section{Attacken gegen WEP}
\begin{frame}
\frametitle{Testing the beamer style}
\begin{itemize}
	\item test
	\item test2
\end{itemize}

\end{frame}


\section{WPA/WPA2(802.11i)}
\begin{frame}
\begin{itemize}
	\item Keines der Schutzziele von WEP (Authentizität, Vertraulichkeit, Integrität) wird erfüllt
	\item[$\Rightarrow$] Entwicklung von 802.11i (WPA/WPA2)
\end{itemize}
\begin{block}{802.11i}
\begin{itemize}
	\item Schlüsselverwaltung:
	\begin{itemize}
		\item Personal/ Pre-Shared Key (PSK)
		\item Enterprise (802.1X)
	\end{itemize}
	\item Sicherheitsprotokolle:
	\begin{itemize}
		\item TKIP(WPA, optional WPA2)
		\item AES-CCMP(optional WPA, WPA2)
	\end{itemize}
\end{itemize}
\end{block}
\end{frame}

\begin{frame}
	\frametitle{Authentifizierung mit 802.1X in 802.11i}
	\begin{itemize}
		\item Verwendung port-basierter Authentifizierung wie im Kabelgebundenen Layer-2
		\item[$\Rightarrow$] Unterstützung für u.a. Zertifikat- und Smartcard-basierter Authentifizierung.
	\end{itemize}
\end{frame}

\begin{frame}
\frametitle{Fehlendes:}
\begin{itemize}
	\item 4-Way Handshake
	\item Schlüsselhierachie
	\item Hole 196
	\item AES-CCMP
\end{itemize}
\end{frame}

\begin{frame}
\frametitle{TKIP (Temporal Key Integrity Protocol)}
\begin{itemize}
	\item AES sollte in 802.11i integriert werden.
	\item AES Berechnungen aufwendig und erforderten neue Hardware
\end{itemize}
\begin{block}{Übergangslösung TKIP}
	\begin{itemize}
		\item Erweiterung von WEP
		\item RC4 weiterhin als Basis
		\item Dynamische statt statischer Schlüssel
		\item Größere IVs:
		\begin{itemize}
			\item 48 bit statt 24 bit
			\item Benutzt als TSC(TKIP Sequence Counter)
		\end{itemize}
		\item Michael (MIC) als verbesserte Intigritätsprüfung
	\end{itemize}
\end{block}
\end{frame}

\begin{frame}
\frametitle{Schwachstellen von TKIP}
\begin{itemize}
	\item Michael hat Schwachstellen (Brute-Force, DoS Angriffe, etc.)
	\item Wörterbuchangriff auf PBKDF2 langsam(4096 Runden HMAC-SHA1)
	\begin{itemize}
		\item Beschleunigung durch Rainbow Tables \cite{renderlab} (SSID + Passphrases)
		\item Berechnung in der Cloud \cite{cloudcracker}: 20 Minuten, 17\$, 300.000.000 Wörter
	\end{itemize}
\end{itemize}
\end{frame}

\section{Andere Attacken}
\subsection{WPS}

\begin{frame}
\frametitle{WPS (Wi-Fi Protected Setup)}
Drei Endnutzerfreundliche WLAN Konfigurationsmodi:
\begin{block}{Push-Button-Connect (“PBC”)}
Benutzer drückt physischen oder virtuellen Knopf.
\end{block}
\begin{block}{PIN - Internal Registrar}
Benutzer gibt mitgelieferten WPS Pin seines Endgerätes in Webmaske des AP ein.
\end{block}
\begin{block}{PIN - External Registrar}
Benutzer bekommt Pin vom Betreiber des APs mitgeteilt und gibt ihn in sein Endgerät ein.
\end{block}
\end{frame}


\begin{frame}
\frametitle{Angriff auf WPS, PIN - External Registrar \cite{wps_attack}}
\begin{itemize}
	\item PIN ist 8 Ziffern lang.
	\item Protokoll erlaubt es zu erkennen ob die ersten 4 Zeichen oder 4 Zeichen falsch waren.
	\item 8. Ziffern Checksumme der ersten 7 Ziffern.
	\item Zu testende Kombinationen: $10^4+10^3 = 11.000$
	\item WPS Standard schreibt kein Lockdown vor. Daher selten implementiert.
	\item Authentification benötigt im Schnitt 1,3 Sekunden.
\end{itemize}
\end{frame}

\subsection{User Tracking}
\begin{frame}
\begin{itemize}
	\item Mülltonnen mit Werbebildschirmen und WLAN erfassten Fussgänger anhand ihrer Handies in London \cite{mulltonnen}
	\item Endgeräte auf der Suche nach versteckten SSIDs broadcasten den Namen der versteckten SSID
	\begin{itemize}
		\item Schlechte Implementationnen broadcasten auch den Namen der unversteckten SSIDs
	\end{itemize}
\end{itemize}
\end{frame}

\section{Häufige Absicherungsempfehlungen}
\begin{frame}
\frametitle{Häufige Absicherungsempfehlungen}
\begin{block}{MAC-Filter}
Kein Nennenswerter Sicherheitsgewinn, da MAC Adresse immer unverschlüsselt übertragen wird.
\end{block}
\begin{block}{SSID Verstecken}
AP broadcastet die SSID zwar nicht, aber sobald ein bekannter Teilnehmer sich anmeldet, wird die SSID unverschlüsselt übertragen.
\end{block} 
\begin{block}{Zufällige SSID}
Wörterbuchattacken gegen PBKDF2 nicht mehr so einfach.
\end{block}
\end{frame}

\section{Tools}
\begin{frame}
\frametitle{Tools}
\begin{block}{Aircrack-ng Suite (siehe \cite{aircrack})}
	\begin{description}
		\item[aircrack-ng] WEP, WPA, WPA2-PSK Key Cracker
		\item[aireplay-ng] Erzeugt WLAN Traffic
		\item[airdrop-ng] Regelbasiertes Deauth Tool
		\item[airodump-ng] Zeichnet WLAN Rohdaten auf
		\item[airdecap-ng] Entschlüsselt aufgezeichnete WEP, WPA, WPA2 Streams
		\item[...]
	\end{description}
\end{block}
\begin{block}{FakeAP (siehe \cite{fakeap})}
Simulieren falscher WLANs
\end{block}
\begin{block}{Kismet (siehe \cite{kismet})/NetStumbler (siehe \cite{netstumbler})}
WLANs aufspüren und kartografieren
\end{block}
\end{frame}

\section{References}
\begin{frame}[allowframebreaks]
\frametitle{References}
\bibliographystyle{IEEEtran}
% argument is your BibTeX string definitions and bibliography database(s)
\bibliography{references}
\end{frame}
\end{document}
