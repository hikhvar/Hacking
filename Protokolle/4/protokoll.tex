\documentclass[10pt,a4paper]{article}
\usepackage[utf8]{inputenc}
\usepackage{amsmath}
\usepackage{amsfonts}
\usepackage{amssymb}
\usepackage{graphicx}
\usepackage{todo}
\usepackage[ngerman]{babel}
\usepackage[left=2cm]{geometry}

% Default fixed font does not support bold face
\DeclareFixedFont{\ttb}{T1}{txtt}{bx}{n}{9} % for bold
\DeclareFixedFont{\ttm}{T1}{txtt}{m}{n}{9}  % for normal

% Custom colors
\usepackage{color}
\definecolor{deepblue}{rgb}{0,0,0.5}
\definecolor{deepred}{rgb}{0.6,0,0}
\definecolor{deepgreen}{rgb}{0,0.5,0}


\usepackage{listings}

% Python style for highlighting
\newcommand\pythonstyle{\lstset{
language=Python,
basicstyle=\ttm,
otherkeywords={self},             % Add keywords here
keywordstyle=\ttb\color{deepblue},
emph={MyClass,__init__},          % Custom highlighting
emphstyle=\ttb\color{deepred},    % Custom highlighting style
stringstyle=\color{deepgreen},
frame=tb,                         % Any extra options here
showstringspaces=false,           % 
numbers=left
}}


% Python environment
\lstnewenvironment{python}[1][]
{
\pythonstyle
\lstset{#1}
}
{}

\lstnewenvironment{bash}
{\lstset{numbers=left,language=bash,keywordstyle={\color{blue}}}}
{}

% Python for external files
\newcommand\pythonexternal[1]{{
\pythonstyle
\lstinputlisting{#1}}}

% Python for inline
\newcommand\pythoninline[1]{{\pythonstyle\lstinline!#1!}}
\lstset{breaklines=true} 
\lstset{basicstyle=\small\ttfamily}
\lstset{framesep=4pt} 

\author{Christoph Robbert 6577945, Peter Stilow 6500440}
\title{Protokoll 4}
\begin{document}
\maketitle
 
\section*{Aufgabe 1}

\subsection*{a)}

Nachdem wir eine Virtuelle Maschine mit der Livelinuxdistribution C.A.IN.E und einem 1 GB großem Arbeitslaufwerk und dem Image des Laptops gestartet hatten legten wir mittels \texttt{Guymager} ein Image des Images des Laptops an. Dieses Image speicherten wir im \texttt{Expert Witness Format}. Danach erstellten wir in Autopsy einen neuen Fall und benutzten dazu das mit \texttt{Guymager} erstellte Image als Image in unserem Fall.

\subsection*{b)}

Es wurde ein Laufwerk C gefunden, auf dem wir zunächst mit der \textit{File Type} Funktion alle Dateien des Systems kategorisieren ließen. Zuerst fiel uns im Ordner\\
\texttt{C:/\$RECYCLE.BIN/S-1-5-21-1799391407-34945392703861017217-1003}\\
eine SQLite Datenbank namens \texttt{\$RVLNSU2.db} auf. Diese kopierten wir mittels der \textit{Export}-Funktion in das Home-Verzeichnis und öffneten sie mit dem mitgelieferten \texttt{SQLite database browser}. In der Datenbank befand sich eine Tabelle namens \texttt{passwords} mit den Spalten \texttt{id} und \texttt{flags}. In dieser Tabelle befanden sich dreißig Flags die dem Format [0-9a-z]{40} entsprachen. Der erste Flag lautet:
\texttt{b6589fc6ab0dc82cf12099c1c2d40ab994e8410c}.

In dem Unterordner \texttt{C:/Pictures} befand sich das Bild \texttt{14fs6f8e46f4se8f48sef66sefsef.jpg}. Nachdem wir das Bild geöffnet hatten, sahen wir einen QR-Code. Leider ließ er sich auf Anhieb nicht öffnen, da zwei der zur Ausrichtung benötigten Quadrate weiß statt schwarz waren. Nachdem wir sie per GIMP wieder hergerichtet hatten, las der QR-Code Reader auf unserem Handy folgenden String aus dem Code aus:
\\\texttt{6c5db859acd48b6be44d02bd48773a2146809629}.

Im selben Ordner befand sich das Bild \texttt{Autumn.jpg}. Der recht auffordernde Satz \textit{Can you find the secret?} in dem Bild lässt vermuten, dass sich in der Datei eine Information im Format \texttt{[0-9a-z]\{40\}} versteckt. Mittels XSteg, einem Tool um verschiedene Steganographie Verfahren aufzuspüren, fanden wir heraus, dass mittels des Programmes \texttt{outguess} etwas in der Datei versteckt wurde. Für weitere Infos zu der Datei benutzten wir das dazugehörige Konsolen-Tool \texttt{steghide} und schauten mit
\begin{verbatim}
steghide info Autumn.jpg
\end{verbatim}
die Datei genauer an. Hiermit fanden wir eine eingebaute Datei \texttt{002\_2.jpg}, die wir mit einem weiteren Befehl
\begin{verbatim}
steghide extract -sf Autumn.jpg
\end{verbatim}
extrahierten. Leider stand dort aber nur \texttt{nice try} drin stand. Wir wiederholten den Vorgang dann sicherheitshalber noch einmal für die gefundene Datei, und siehe da eine weitere Datei \texttt{003\_1.jpg} befand sich darin. In dieser fanden wir dann auch einen passenden flag: \texttt{7e1037f7e2be528d16a0aa657ffb6bc1cd57a26b}.

Auch im Ordner \texttt{Documents/Literature} wurden wir fündig. In einem pdf als Anmerkung versteckt (unten auf Seite 16) fanden wir das folgende flag:
\texttt{044d52b80dc3b1d952ca95f14a96c46c005ab9cc}

Als letztes durchsuchten wir den Ordner \texttt{Music}. Uns fiel eine mit secret benannte Datei (\texttt{CANYON.MID:secret}) auf, in der eine Zeichenkette im ASCII Format vorlag, allerdings nicht passend zu unserem flag Format. Durch viel herumprobieren und umkonvertieren kamen wir zu dem Ergebnis, dass eine Base64 Codierung vorliegt. Also decodierten wir die Zeichenkette zu UTF-8 und bekamen das flag:\\
\texttt{8f8780bdc9d31ba2cae506f2a629978b915bddff}

\section*{Aufgabe 2}
Wir benannten als erstes die Datei \texttt{93f5c9434cd3b29ffdfb986d993ba469f96ff6ec} in \texttt{crash\_dump} um.
Um das Tool \texttt{volatility} zu verwenden muss man wissen von welcher Windowsversion das Speicherabbild stammt. Da dieses nicht gegeben ist, versuchen wir es mittels des \texttt{volatility} Kommandos \texttt{imageinfo} herrauszufinden um was für ein Speicherabbild es sich handelt.
\begin{verbatim}
python vol.py imageinfo -f crash_dump
Volatility Foundation Volatility Framework 2.3.1
Determining profile based on KDBG search...

          Suggested Profile(s) : Win7SP0x86, Win7SP1x86
                     AS Layer1 : IA32PagedMemoryPae (Kernel AS)
                     AS Layer2 : VirtualBoxCoreDumpElf64 (Unnamed AS)
                     AS Layer3 : FileAddressSpace (/home/christoph/Downloads/crash_dump)
                      PAE type : PAE
                           DTB : 0x185000L
                          KDBG : 0x82930c28
          Number of Processors : 4
     Image Type (Service Pack) : 1
                KPCR for CPU 0 : 0x82931c00
                KPCR for CPU 1 : 0x807c1000
                KPCR for CPU 2 : 0x88840000
                KPCR for CPU 3 : 0x88876000
             KUSER_SHARED_DATA : 0xffdf0000
           Image date and time : 2013-12-16 17:04:54 UTC+0000
     Image local date and time : 2013-12-16 18:04:54 +0100
\end{verbatim}
Die Ausgabe legt nahe, dass es sich um ein 32-Bit Windows 7 mit entweder Servicepack 1 oder 2 handelt. Da die Kommandos
\begin{verbatim}
python vol.py -f crash_dump --profile=Win7SP1x86 kdbgscan
\end{verbatim}
und 
\begin{verbatim}
python vol.py -f crash_dump --profile=Win7SP0x86 kdbgscan
\end{verbatim}
keine Unterscheidbare Ausgabe lieferten, gingen wir davon aus, dass es sich um ein Windows 7 mit SP1 handelt.

Um einen groben Überblick zu erlangen zeigten wir uns mit 
\begin{verbatim}
python vol.py -f crash_dump --profile=Win7SP0x86 psscan
\end{verbatim}
alle laufenden und beendeten Prozesse an. Dabei vielen die Prozesse
\begin{verbatim}
Offset(P)  Name                PID   PPID PDB        Time created                   
---------- ---------------- ------ ------ ---------- ------------------------------
..
0x1eaecd40 notepad.exe        2628   1536 0x1f73e420 2013-12-16 16:49:32 UTC+0000
..
0x1f7b53f0 WUDFHost.exe       2864    876 0x1f73e480 2013-12-16 16:57:56 UTC+0000
..
0x1fd79758 ftp.exe             124    328 0x1f73e560 2013-12-16 17:02:59 UTC+0000
\end{verbatim}
im Hinblick auf die nächsten Aufgabenteile auf. 

\subsection*{a)}
Der schon aufgefallene Prozess \texttt{WUDFHost.exe} ist ein Userspace Treiber Prozess. Da USB Sticks häufig per Userspace Treiber angesprochen werden, schauten wir uns die offenen Handles dieses Prozesses per 
\begin{verbatim}
python vol.py -f crash_dump --profile=Win7SP1x86 handles --pid=2864
\end{verbatim}
an. Dabei vielen vor allem die Handles auf die dieser Registrykey auf.
\begin{lstlisting}
0xa55584b8   2864      0x1e4    0x20019 Key              MACHINE\SYSTEM\CONTROLSET001\ENUM\WPDBUSENUMROOT\UMB\2&37C186B&0&STORAGE#VOLUME#_??_USBSTOR#DISK&VEN_KINGSTON&PROD_DATATRAVELER_G3&REV_PMAP#001CC0EC345CBA9175C40135&0#\DEVICE PARAMETERS
\end{lstlisting}
Dabei fallen die drei Substrings  \texttt{STORAGE}, \texttt{USBSTOR}, \texttt{VEN\_KINGSTON} auf. Besonders der letzte String lässt auf den Hersteller (Vendor) Kingston schließen. Direkt hinter dem Substring \texttt{VEN\_KINGSTON} fällt der Substring \texttt{PROD\_DATATRAVELER\_G3} auf. Eine Googlesuche nach \texttt{DATATRAVELER\_G3} fördert einen Kingston USB Stick mit dem Namen \texttt{Kingston Datatraveler G3} hervor. Es handelt sich also sehr wahrscheinlich um diesen USB Stick, den Bernd verwendet hat.

\subsection*{b)}

Der Aufruf
\begin{verbatim}
python vol.py -f crash_dump --profile=Win7SP1x86 handles --pid=2628
\end{verbatim}
zeigte keinen offenen Filehandles für den gestarteten notepad.exe Prozess.
Nachdem wir mittels
\begin{verbatim}
python vol.py -f crash_dump --profile=Win7SP1x86 memdump --pid=2628 --dump-dir=output
\end{verbatim}
erzeugten wir ein Memorydump des Prozesses notepad.exe. 
Danach erzeugten wir eine String Repräsentation des Dumpes mittels
\begin{verbatim}
strings output/2628.dmp > output/2628.txt
\end{verbatim}
Diese Datei \texttt{2628.txt} durchsuchten wir nach auffälligen Strings. Wir fanden folgenden String, der nach einem Programmaufruf des Notepads aussah 
\begin{verbatim}
"C:\Windows\system32\NOTEPAD.EXE" C:\Users\Bernd\Documents\secret_pw.txt.txt
\end{verbatim}
und weitere Hinweise darauf, das der zuvor mittels \texttt{notepad.exe} die Datei \\\texttt{C:\textbackslash Users\textbackslash Bernd\textbackslash Documents\textbackslash secret\_pw.txt.txt} bearbeitet wurde.


\subsection*{c)}
Um das Volatility Plugin \texttt{cryptoscan} zu benutzen luden wir uns die Volatility Version 1.3 herunter. Anschließend starteten wir mit diesem Modul ein Scan nach dem Passwort.
\begin{verbatim}
python volatility cryptoscan -f crash_dump
\end{verbatim}
Als Ausgabe erhielten wir folgendes:
\begin{verbatim}
2106632:uint64
26956940:HELPDIR
29181068:TypeLib
80299140:v!6.1.7601.18027
83534988:00000009
87614596:7000000008525
88153652:D 
154371212:Indexes
184935564:6.1
215250052:v!6.1.7601.17750
268936324:InprocServer32
297604236:00000000
324158604:24D
331007116:00000000
332280940:{0000061D-0000-0010-8000-00AA006D2EA4}
337179756:{071D41B6-8806-4EB0-B661-6CB67BE6E86E}
342893708:00000000
350016652:Indexes
376972412:f256!acxtrnal.dll
379982964:Microsoft.DirectMusicScript.1
412510104:InvalidFlag
413942192:normalized
429233260:{05AF94D8-7174-4CD2-BE4A-4124B80EE4B8}
434017412:Properties
440579204:v!6.1.7600.16385
446350476:Dhcp
470616396:D 
489993464:7h15_p455w0rd_15_54f3!?
\end{verbatim}
Wir schlossen aus der in Leetspeak formatierten letzten Zeile, dass dies das Passwort ist.
Das Truecryptpasswort lautet also:
\texttt{7h15\_p455w0rd\_15\_54f3!?}.

\subsection*{d)}
Wie oben schon zu sehen war ein \texttt{ftp} Prozess geöffnet. Mittels
\begin{verbatim}
python vol.py -f crash_dump --profile=Win7SP1x86 memdump --pid=124 --dump-dir=output
\end{verbatim}
und \texttt{strings}
schauten wir uns auch diesen Memorydump als Strings an. Dabei fielen uns diese Zeilen auf.
\begin{verbatim}
password.
3GonELQR
open
192.26.175.74
[..]
USER wernerb
[..]
PASS 3GonELQR
\end{verbatim}
Mittels dem Benutzer \texttt{wernerb} und dem Passwort \texttt{3GonELQR} konnten wir uns dann tatsächlich auf dem Host \texttt{192.26.175.74} per FTP einloggen. Dies sind also sehr wahrscheinlich die gesuchten Credentials.
\end{document}