\documentclass{beamer}
\usetheme[footline=infoline,headline=secheader]{UPB}
%\usetheme[footline=infoline,headline=structure]{UPB}
%\usetheme[headline=secheader]{UPB}
%\usetheme[headline=structure]{UPB}
%\usetheme[footline=infoline]{UPB}
%\usetheme{UPB} % defaults are footline=empty,headline=empty
\usepackage[utf8]{inputenc}
\author[C.Robbert, P. Stilow]{Christoph Robbert, Peter Stilow}
\institute[Uni Paderborn]{Universität Paderborn}
\title[WLAN Security]{WLAN Security}
\begin{document}
\begin{frame}
\maketitle
\end{frame}
\section{Attacken gegen WEP}
\begin{frame}
\frametitle{Testing the beamer style}
\begin{itemize}
	\item test
	\item test2
\end{itemize}
\end{frame}

\section{Attacken gegen WPA/WPA2}

\section{Andere Attacken}
\subsection{WPS}

\begin{frame}
\frametitle{WPS (Wi-Fi Protected Setup)}
Drei Endnutzerfreundliche WLAN Konfigurationsmodi:
\begin{block}{Push-Button-Connect (“PBC”)}
Benutzer drückt physischen oder virtuellen Knopf.
\end{block}
\begin{block}{PIN - Internal Registrar}
Benutzer gibt mitgelieferten WPS Pin seines Endgerätes in Webmaske des AP ein.
\end{block}
\begin{block}{PIN - External Registrar}
Benutzer bekommt Pin vom Betreiber des APs mitgeteilt und gibt ihn in sein Endgerät ein.
\end{block}
\end{frame}


\begin{frame}
\frametitle{Angriff auf WPS, PIN - External Registrar\footnote{http://sviehb.files.wordpress.com/2011/12/viehboeck\_wps.pdf}}
\begin{itemize}
	\item PIN ist 8 Ziffern lang.
	\item Protokoll erlaubt es zu erkennen ob die ersten 4 Zeichen oder 4 Zeichen falsch waren.
	\item 8. Ziffern Checksumme der ersten 7 Ziffern.
	\item Zu testende Kombinationen: $10^4+10^3 = 11.000$
	\item WPS Standard schreibt kein Lockdown vor. Daher selten Implementiert.
	\item Authentification benötigt im Schnitt 1,3 Sekunden.
\end{itemize}
\end{frame}

\section{Häufige Absicherungsempfehlungen}
\begin{frame}

\begin{block}{MAC-Filter}
Kein Nennenswerter Sicherheitsgewinn, da MAC Adresse immer unverschlüsselt übertragen wird.
\end{block}
\begin{block}{SSID Verstecken}
AP broadcastet die SSID zwar nicht, aber sobald ein bekannter Teilnehmer sich anmeldet, wird die SSID unverschlüsselt übertragen.
\end{block}
\end{frame}


\end{document}
