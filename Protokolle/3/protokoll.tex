\documentclass[10pt,a4paper]{article}
\usepackage[utf8]{inputenc}
\usepackage{amsmath}
\usepackage{amsfonts}
\usepackage{amssymb}
\usepackage{graphicx}
\usepackage{todo}
\usepackage[ngerman]{babel}
\usepackage[left=2cm]{geometry}

% Default fixed font does not support bold face
\DeclareFixedFont{\ttb}{T1}{txtt}{bx}{n}{9} % for bold
\DeclareFixedFont{\ttm}{T1}{txtt}{m}{n}{9}  % for normal

% Custom colors
\usepackage{color}
\definecolor{deepblue}{rgb}{0,0,0.5}
\definecolor{deepred}{rgb}{0.6,0,0}
\definecolor{deepgreen}{rgb}{0,0.5,0}


\usepackage{listings}

% Python style for highlighting
\newcommand\pythonstyle{\lstset{
language=Python,
basicstyle=\ttm,
otherkeywords={self},             % Add keywords here
keywordstyle=\ttb\color{deepblue},
emph={MyClass,__init__},          % Custom highlighting
emphstyle=\ttb\color{deepred},    % Custom highlighting style
stringstyle=\color{deepgreen},
frame=tb,                         % Any extra options here
showstringspaces=false,           % 
numbers=left
}}


% Python environment
\lstnewenvironment{python}[1][]
{
\pythonstyle
\lstset{#1}
}
{}

\lstnewenvironment{bash}
{\lstset{numbers=left,language=bash,keywordstyle={\color{blue}}}}
{}

% Python for external files
\newcommand\pythonexternal[1]{{
\pythonstyle
\lstinputlisting{#1}}}

% Python for inline
\newcommand\pythoninline[1]{{\pythonstyle\lstinline!#1!}}


\author{Christoph Robbert 6577945, Peter Stilow 6500440}
\title{Protokoll 3}
\begin{document}
\maketitle
 
\section*{Aufgabe 1}
\subsection*{a)}

Um das Passwort von bernd (bzw des root users) des Systems zu ändern muss man sich nur über eine live-Version von linux auf den rechner einloggen und hat komplette Kontrolle. Hier kann man z.B. das Passwort des ursprünglichen Systems ändern. In unserem Fall haben wir Arch-Linux benutzt:

Zunächst laden wir die Arch-Linux installations iso herunter. Diese legen wir in VirtualBox als CD ein (bei einem realen pc hätte man das natürlich per USB-Stick lösen können). Von dieser ISO booten wir also den PC von Bernd Power. Wir kommen in die normale Arch-Linux Konsole für die Installation. Anstatt zu installieren mounten wir die Festplatten (/dev/sda*), wobei sda3 die gewünschte root Partition und sda4 bernds home Verzeichnis ist (bzw alle home Verzeichnisse). Wir mounten also via:

mount /dev/sda3 /mnt
mount /dev/sda4 /mnt/homes

Jetzt wechseln wir den root in das gemountete Verzeichnis (von Arch-Linux in eine Arch-Linux installation wechseln können wir den arch-chroot Befehl benutzen):

arch-chroot /mnt

Um das root Passwort zu wechseln reicht jetzt ein passwd befehl mit 2-maliger Bestätigung des neuen Passwortes. Für den user bernd benutzen wir:

passwd bernd

und setzen in unserem falle das Passwort auf bernd. Um Bernd Administratorrechte zu geben (wir könnten natürlich alternativ auch direkt per root alles machen in Bernds System) entscheiden wir uns für den Eintrag in die Gruppe "`wheel"', damit Bernd per sudo Befehle als admin durchführen kann:

usermod -aG wheel bernd

Danach musste noch die wheel-Gruppe freigeschaltet werden:

visudo

In dem erscheinenden Editor musste folgende Zeile einkommentiert werden:

\# %wheel ALL=(ALL) NOPASSWD:ALL

Durch Neustarten des Systems und booten der original-Installation können wir uns jetzt einfach als root oder bernd mit dem passwort bernd anmelden.

\subsection*{b)}

Zum finden der versteckten Passwörter benutzen wir in der Hauptsache den find Befehl verknüpft mit grep, wie z.B. im home Verzeichnis von Bernd:

find . | xargs grep "[0-9]\{32\}" -sl

Durch die ausgegebenen Dateien und weitere manuelle Suche stoßen wir auf folgende Passwörter/Flags:

\begin{verbatim}
~/.purple/logs/jabber/berndbernd@jabber.ccc.de/berndberndbernd@jabme.de/
2013-05-09.204357+0200CEST.html
[...] das neue passwort für den dienst ist
928c625b8ded159423369a116ac28f79
\end{verbatim}

\begin{verbatim}
~/.filezilla/sitemanager.xml
<Pass>
8103c4097fecbb013b3243658efb7c0e
</Pass>
\end{verbatim}

\begin{verbatim}
~/.irssi/.logs/freenode/#hacksem/messages.log
Zur Info das vorläufige Passwort:
6c790e02a808602886b32b7151dac514
\end{verbatim}

\begin{verbatim}
~/.subversion/auth/svn.simple/62891e6589fd019e0ca49c76607c64c4
password V 32
d63036cfba56a6c3cd75ab14b33fe67g
\end{verbatim}

\begin{verbatim}
~/.mozilla/thunderbird/bcx3c2rl.default/ImapMail/imap.bernd.bernd/INBOX
hier ist das Passwort, das du neulich so dringend haben wolltest.
4b6917426d1481d401g42faa7e67acf4
\end{verbatim}

Auch in dieser verdächtigen Datei wird sich vermutlich ein Passwort befinden!

\begin{verbatim}
~/bernds_pw.pl
?!?
\end{verbatim}

Weitere Funde des Flag-formats (jedoch ohne Hinweise auf Passwörter) waren unter anderem (in):
\begin{verbatim}
dateiname
~/.suberversion/auth.svn.ssl.server/574aef086c4a5beafd1e16043de3a508
\end{verbatim}

\begin{verbatim}
~/.dbus/session-bus/772ccb2f8e93d8bae40f7a9e0000012b-1
guid=
3b590b86735ccb66e0d982400000015f
\end{verbatim}

\begin{verbatim}
~/.dbus/session-bus/772ccb2f8e93d8bae40f7a9e0000012b-1
guid=
4996f8c8ba8c1bf505c1cccc00000009
\end{verbatim}

\section*{Aufgabe 2}

Gegen die in Zip eingesetzte PkZip Verschlüsselung gibt es einen \textit{known plaintext} Angriff. Das Tool \texttt{pkcrack} führt diese Attacke automatisch auf Zipdateien aus, wenn man eine der enthaltenen Dateien im Klartext vorliegen hat. Da man die Zipdatei öffnen kann und sich die enthaltenen Dateinamen anschauen kann, fanden wir die folgenden Dateien.
\begin{verbatim}
Brace-yourself-Winter.jpg
642dfff0ebcf31b55400bd205bfdfa06bfb1768c.jpg
\end{verbatim}

Da wir für den Angriff eine Datei im Klartext brauchten führten wir eine Google Bildersuche durch. Für das Bild \texttt{Brace-yourself-Winter.jpg} fanden wir ein Bild, dass genauso groß war wie das verschlüsselte. Da \texttt{pkcrack} eine komprimierte Variante des Klartextes benötigt, erstellten wir ein komprimiertes Zipfile \texttt{brace.zip} mit dem Linuxkommandozeilentool \texttt{zip.} Dies riefen wir wie folgt auf:
\begin{verbatim}
zip brace.zip Brace-yourself-Winter.jpg
\end{verbatim}

Nun hatten wir alle benötigten Eingaben für \texttt{pkcrack} zusammen. Wir starteten \texttt{pkcrack} also mit den Parametern:
\begin{small}
\begin{verbatim}
pkcrack -C packed.zip -c Brace-yourself-Winter.jpg -P brace.zip -p Brace-yourself-Winter.jpg -d solution.zip -a
\end{verbatim}
\end{small}

Dieser Aufruf wandte den Angriff automatisch auf das Archiv \texttt{packet.zip} an und schrieb eine Entschlüsselte Variante des Archivs in die Datei \texttt{solution.zip}. Die brisanten Informationen sind ein Bild, dass diverse Bücher von George R.R. Martin zeigt und den String \texttt{84a940b74b33ac6e040ab0462fb9a6f9} als Text im Bild enthält.


\section*{Aufgabe 3}

Diese Aufgabe haben wir mittels der Programmiersprache Python gelöst. In der Standardlibrary von Python ist eine Implementierung der Hashfunktion SHA256 enthalten. Wir nutzen diese um die Hashes zu berechnen. 

Im Folgenden beschreiben wir die wesentlichen Eigenschaften des in Abbildung \ref{pyaufgabe3} abgebildeten Sourcecodes. Die Funktion \texttt{read\_file} liest die Datei ein, sorgt dabei dafür, dass die Strings in UTF-8 codiert bleiben und gibt das Ergebnis als Liste von Worten zurück. Die UTF-8 Codierung ist wichtig, da ein berechnen der Hashes auf der ASCII Codierung der Strings zu einem anderen Hashwert führt. Die Funktion \texttt{test\_all} führt dann einen Brute-Force Angriff auf den gegeben Hashwert mittels der gegeben Wortlisten durch. Die drei \texttt{for}-Schleifen erzeugen alle Kombinationen aus den Wortlisten und in Zeile 34 wird dann geschaut, ob die aktuelle Kombination den gesuchten Hash ergeben. Um die Berechnung effizienter zu gestalten wird der aktuelle Hashwert \texttt{m,m1,m2} immer nur mit dem neu hinzugekommen Wort geupdated. 

Ein handelsüblicher Laptop berechnet mit diesem Programm die folgende Ausgabe innerhalb von 2 Minuten:
\begin{verbatim}
Erstes Passwort:  BossBasketballKonfigurierbarkeit
Zweites Passwort:  KombattantCoupeBereifen
\end{verbatim} 

\begin{figure}
\center
\caption{Python Sourcecode für die Aufgabe 3}
\label{pyaufgabe3}
\pythonexternal{python/aufgabe3.py}
\end{figure}



\end{document}